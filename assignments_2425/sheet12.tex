%%%%%%%%%%%%%%%%%%%%%%%%%%%%%%%%%%%%%%
% Numerical Linear Algebra class 2022 
% Sheet 12
%%%%%%%%%%%%%%%%%%%%%%%%%%%%%%%%%%%%%%

\begin{Sheet}[to be handed in until January 31, 2024, 11am.]
  \label{sheet12}

  The following problems review some topics of this lecture.
They are a good preparation for the oral exam and they will also help with better understanding the programming problem and project.
Please summarize your thoughts on those using bullet points.

  \begin{Problem}
	Householder reflections vs. Givens rotations: which one is more
	cost-efficient in the general case? When using the other one is
	advantageous?
\end{Problem}

\begin{Problem}
	Which of the discussed methods for solving eigenvalue problems can
	be implemented without explicitly forming a matrix?
\end{Problem}

\begin{Problem}
	Consider an $n\times n$ matrix that has $n$ distinct eigenvalues
	such that $\abs{\lambda_i} \neq \abs{\lambda_j}$ for $i\neq
	j$. How can the eigenvalue with the second largest absolute value
	be computed?
\end{Problem}

\begin{Problem}
	When does one step of the Gauss-Seidel iteration provide the
	direct solution of a linear system? Consider an upper triangular
	matrix to visualize this matter.
\end{Problem}

\end{Sheet}


%%% Local Variables: 
%%% mode: latex
%%% TeX-master: "main"
%%% End: 
