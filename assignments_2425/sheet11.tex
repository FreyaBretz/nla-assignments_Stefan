%%%%%%%%%%%%%%%%%%%%%%%%%%%%%%%%%%%%%%
% Numerical Linear Algebra class 2022 
% Sheet 11
%%%%%%%%%%%%%%%%%%%%%%%%%%%%%%%%%%%%%%

\begin{Sheet}[to be handed in until January 24, 2024, 11am.]
  \label{sheet11}
  
\begin{Problem}
  	Use your implementation of Problem 2(a) on Sheet 9 and 
  	augment it such that you can compute the eigenvalues of the projected matrix $\matH_m$ in each step.
  	For the computation of the eigenvalues you could, for example, use your implementation of the QR iteration (Problem 4 on Sheet 5).
  	Test your implementation with $\matl_2$ as in Problem 2(b) on Sheet 9 and observe the convergence of extremal eigenvalues $\lambda_{\text{min}}$ and $\lambda_{\text{max}}$.
\end{Problem}
  
  The following problems review some topics of this lecture.
  They are a good preparation for the oral exam and they will also help with better understanding the programming problem and project.
  Please summarize your thoughts on those using bullet points.

\begin{Problem}
	Recapitulate the concept of an orthogonal projection and an
	oblique projection. What are use cases of both and why are they
	important for numerical linear algebra?
\end{Problem}

\begin{Problem}
	What methods presented in the lecture can be used for computing or
	estimating the eigenvalues of a matrix $\mata$? Sort by properties
	of the methods, as well as by conditions on $\mata$.
\end{Problem}

\begin{Problem}
	How can the QR factorization of a given matrix be computed?
	Discuss downsides and benefits of the different methods.
\end{Problem}

\end{Sheet}


%%% Local Variables: 
%%% mode: latex
%%% TeX-master: "main"
%%% End: 
