%%%%%%%%%%%%%%%%%%%%%%%%%%%%%%%%%%%%%%
% Numerical Linear Algebra class winter semester 2024/2025
% Sheet 1                             
%%%%%%%%%%%%%%%%%%%%%%%%%%%%%%%%%%%%%%

\begin{Sheet}[to be handed in until October 23, 2024, 11am.]
  \label{sheet1}
  
  Please hand in your solutions in groups of 2-4. Solutions submitted by only one person will not be graded.
  The solutions for non-programming problems may be either handed in online via Moodle or on paper (Mathematikon, 1st floor, letterbox to be determined). 
  The solutions for programming problems have to be handed in online via Moodle. Specifics may be found in the respective problems.
  To recieve the points for the exercises, write everyones first and last name on every hand-in. 
  Details will be discussed in the exercise group on Thursday. 
 
  If you currently don’t have access to Moodle, please contact your tutor Freya in the exercise group or via Mail (freya.jensen[at]iwr.uni-heidelberg.de). 
  
\begin{Problem}
  Review the following items and write down at least one of the
  definitions and one of the theorems with proof in detail.
  \begin{itemize}
  \item Definition of a projection
  \item Definition of an orthogonal projection
  \item Theorem: Orthogonal projection is uniquely determined by
    subspace

    Consider a finite-dimensional space $V$ with inner product
    $\scal(\cdot,\cdot)$ and a subspace $W\subset V$. Then, there
    exists a unique orthogonal projection
    \begin{gather*}
       P_W:V\to W.
    \end{gather*}
  \item Theorem: Best Approximation Theorem

    Let $W$ be a subspace of $\R^n$, let $\vx$ be any vector in
    $\R^n$, and let $\tilde{\vx}$ be the orthogonal projection of
    $\vx$ onto $W$. Then $\tilde{\vx}$ is the closest point in $W$ to
    $\vx$, in the sense that
    \begin{gather*}
      \norm{\vx - \tilde{\vx}} < \norm{\vx - \vw}
    \end{gather*}
    for all $\vw$ in $W$ distinct from $\vx$.
  \item Theorem: Orthogonal projection in orthonormal basis
    
    Let $B = \{\vu_1,\vu_2, ...,\vu_p\}$ be an orthonormal basis of a
    subspace $W$ of a finite-dimensional space $V$ with inner product
    $\scal(\cdot,\cdot)$. Then, the orthogonal projection $P_i$ of any
    vector $\vv\in V$ onto $\vu_i$, and the orthogonal projection
    $P_W$ of any vector $\vv\in V$ onto $W$ have the following
    expressions, respectively:
    \begin{gather*}
      P_i(\vv) = \scal(\vv, \vu_i) \vu_i,\qquad i = 1, 2, ..., p,
      \\
      P_W(\vv) = \sum_{i=1}^p \scal(\vv, \vu_i) \vu_i,
    \end{gather*}
    and
    \begin{gather*}
      \vv = P_W(\vv) + \vz,\qquad \vz \perp W.
    \end{gather*}

  \item Theorem: Parseval identity

    Suppose that $W$ is a finite-dimensional linear space with inner
    product $\scal(\cdot,\cdot)$. Let $\{\ve_i\}$, $i=1,...,n$ be an
    orthonormal basis of $W$. Then, for every $\vw\in W$ it holds
    \begin{gather*}
      \sum_{i=1}^n \abs{\scal(\vw,\ve_i)}^2
      = \norm{\vw}^2.
    \end{gather*}
  \end{itemize}
\end{Problem}

\begin{Problem}[Programming]
    To familiarize yourself with the C++ library \href{https://arma.sourceforge.net/}{Armadillo} write a function \texttt{gauss\_eliminiation} that performs the Gaussian elimination for a system of linear equations $Ax = b$.
    You should use the armadillo matrix class and vector class to store the system of linear equations, handle the result and perform the vector matrix multiplication in the end.
    Write a \texttt{main} function in which you test your implementation with the following system of linear equations:
	\[
	\begin{array}{rcrcrcrcr}
		2x_1 & + & 8x_2 & + & 10x_3 & + & 10x_4 & = & 0\\
		x_1 & + & 5x_2 & + & 2x_3 & + & 9x_4 & = & 1\\
		-3x_1 & - & 10x_2 & - & 21x_3 & - & 6x_4 & = & -4\\
		-2x_1 & - & 3x_2 & - & 4x_3 & -& 7x_4 & = & -1
	\end{array}
	\]
    In order to test the correctness of your result you should perform a matrix vector multiplication to see whether the original matrix $A$ multiplied with the resulting vector $x$ yields the vector $b$.
    We will also discuss this problem in the programming tutorial on Thursday.

\end{Problem}

\end{Sheet}


%%% Local Variables: 
%%% mode: latex
%%% TeX-master: "main"
%%% End: 
