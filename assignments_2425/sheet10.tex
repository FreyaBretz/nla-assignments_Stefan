%%%%%%%%%%%%%%%%%%%%%%%%%%%%%%%%%%%%%%
% Numerical Linear Algebra class 2022 
% Sheet 10
%%%%%%%%%%%%%%%%%%%%%%%%%%%%%%%%%%%%%%

\begin{Sheet}[to be handed in until January 17, 2024, 11am.]
  \label{sheet10}

  \begin{Problem}
    Show how GMRES (Algorithm 3.4.42 in the lecture notes) and Arnoldi
    with modified Gram-Schmidt (Algorithm 3.4.10 in the lecture notes)
    will converge on the linear system $\mata\vx=\vb$ when
    \begin{gather*}
      \mata =
      \begin{pmatrix}
        &&&&1\\
        1&&&&\\
        &1&&&\\
        &&1&&\\
        &&&1&
      \end{pmatrix},
      \quad
      \vb =
      \begin{pmatrix}
        1\\ 0\\ 0\\ 0\\ 0
      \end{pmatrix},
    \end{gather*}
    and $\vx_0 = \boldsymbol 0$.
  \end{Problem}

  \begin{Problem}
	Let $\mata$ and $\matb$ be symmetric positive-definite matrices in
	$\R^{n\times n}$. Let there be constants $c_1,c_2$ such that
	\begin{gather*}
	c_1 \vx^T\mata\vx \leq \vx^T\matb\vx \leq c_2 \vx^T\mata\vx.
	\end{gather*}
	Derive an estimate for the convergence of the conjugate gradient
	iteration for $\mata\vx=\vb$ with preconditioner $\matb^{-1}$
	depending on $c_1$ and $c_2$.
	
	\textit{Hints:} Lemma 3.4.68 in the lecture notes, and Thereom
	2.2.17 (Courant-Fischer min-max theorem) (note, that the definition of the Rayleigh quotient is
	independent of the choice of the inner product).
\end{Problem}

\begin{Problem}
	Consider again matrix $\matt_{\alpha}$ as introduced in \cref{sheet6:problem4} on sheet \cref{sheet6}.
	\begin{enumerate}[(a)]
		\item Implement a method that computes the smallest eigenvalue of the matrix 
		$\matt_{\alpha}$ using the inverse iteration (Algorithm 2.6.1 in the
		lecture notes). You may reuse your code from \cref{sheet6:problem5} on sheet \cref{sheet6}. Do not calculate the inverse explicitly for
		solving the appearing linear system, but use the conjugate
		gradient iteration (Algorithm 3.4.28 in the lecture notes) for
		this matter.
		\item Use your implementation to calculate the smallest eigenvalue
		of $\matt_{\alpha}$ with $\alpha=2$ and $n=20$.
	\end{enumerate}
\end{Problem}

  \vfill
  \bibliographystyle{alpha}
  \bibliography{bib}
\end{Sheet}


%%% Local Variables: 
%%% mode: latex
%%% TeX-master: "main"
%%% End: 
