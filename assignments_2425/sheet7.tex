%%%%%%%%%%%%%%%%%%%%%%%%%%%%%%%%%%%%%%
% Numerical Linear Algebra class 2022 
% Sheet 7
%%%%%%%%%%%%%%%%%%%%%%%%%%%%%%%%%%%%%%

\begin{Sheet}[to be handed in until December 4, 2024, 11am.]
  \label{sheet7}

  \begin{Problem}
    Prove Theorem 3.2.15 in the Lecture Notes (Convergence of Richardson Iteration).
  \end{Problem}

  \begin{Problem}
    Problem B.3.13 in the lecture notes.
  \end{Problem}

  \begin{Problem}[Programming]
    \label{sheet7:problem3}
    Let $\matl_d$ be the discretization of the $d$-dimensional Laplace
    operator on the unit square by the five-point stencel with a
    uniform Dirichlet boundary condition to the mesh size
    $h=\frac1{n+1}$.  In 1D it is given in terms of the matrix
    $\matt_2$ defined in \cref{sheet6:problem4} of \cref{sheet6} by
    \begin{gather*}
    \matl_1 = \frac1{h^2}\matt_2 \in\R^{n\times n}.
    \end{gather*}
    In 2D it is given by
    \begin{gather*}
    \matl_2 = \frac1{h^2}
    \begin{pmatrix}
    \matd &   -\id & & & \\
    -\id &  \matd & -\id   & & \\
    & \ddots & \ddots & \ddots & \\
    &        &   -\id &  \matd &   -\id\\
    &        &        &   -\id &  \matd
    \end{pmatrix}
    \in\R^{n^2\times n^2},
    \end{gather*}
    where $\id\in\R^{n\times n}$ and $\matd\in\R^{n\times n}$ are
    defined as
    \begin{gather*}
    \id =
    \begin{pmatrix}
    1 & & \\
    & \ddots & \\
    &        & 1
    \end{pmatrix}
    \quad\text{and}\quad
    \matd =
    \begin{pmatrix}
    4 & -1 &    && \\
    -1 &  4 & -1 && \\
    & \ddots & \ddots & \ddots & \\
    &    & -1 & 4 & -1 \\
    &    &    & -1 & 4
    \end{pmatrix}.
    \end{gather*}
    
    Details may be found in appendix B.3 on finite difference methods in the lecture notes. 
    
    Let $\mata=\matl_2\in\R^{n^2\times n^2}$ as defined above and $\vb\in \R^{n^2}$.
    Write a program that solves the 2D Laplace problem
    \begin{gather*}
      \mata\vx=\vb,
    \end{gather*}
    with the Richardson iteration by observing the following
    steps:
    \begin{enumerate}[(a)]
    \item\label{sheet7:problem4:parta} Implement a function
      \lstinline{vmult} that performs the matrix-vector product
      $\mata\vv$ of $\mata$ with a given vector $\vv$, and
      returns a vector $\vw$. 
      Do not store the system matrix in this function.
      The parameters should only be the vector $\vv$ and, optionally, $n$.
    \item Write a function \lstinline{richardson_step} that
      implements a single step of the Richardson iteration. Similar to 
      \eqref{sheet7:problem4:parta} this function should take vectors
      $\vv$ and $\vb$, optionally $n$, and return the resulting vector $\vw$.
      \textit{Hint: The eigenvalues of $\matl_2$ are given by the pairwise sums of the eigenvalues of $\matt_2$, see B.3.13.}
    \item Use the null-vector $\vx^{(0)} = (0,...,0)^T$ as initial
      guess and test your program for the constant vector $\vb=(1,...,1)^T$ with $n=20$ and $n=100$.
      Observe the evolution of the residual in the 2-norm
      \begin{gather*}
        r^{(k)} = \norm{\mata\cdot\vx^{(k)}-\vb}_2
      \end{gather*}
      for 50 steps of the Richardson iteration.  
   \item Plot the obtained
      result vector after 1, 5, 15, 30, and 50 iterations as a two-dimensional function.
      \textit{Hint: For plotting you have to invert the numbering of the vector (use, for example, \lstinline{np.reshape}) and then plot the two dimensional grid using a numpy color map (greyscale).}      
  \item Compute and discuss the observed convergence rate (see Definition 3.2.22 in the Lecture Notes).
   \end{enumerate}
  \end{Problem}

\end{Sheet}


%%% Local Variables: 
%%% mode: latex
%%% TeX-master: "main"
%%% End: 
