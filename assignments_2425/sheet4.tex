%%%%%%%%%%%%%%%%%%%%%%%%%%%%%%%%%%%%%%
% Numerical Linear Algebra class 2022 
% Sheet 4
%%%%%%%%%%%%%%%%%%%%%%%%%%%%%%%%%%%%%%

\begin{Sheet}[to be handed in until November 15, 2023,11am.]
  \label{sheet4}



  \begin{Problem}
  	Write a program \texttt{COMPUTE\_EIGENVALUES} that
  	\begin{enumerate}
  		\item computes the dominant eigenvalue of a matrix using the power method.
  		\item computes all further eigenvalues using suitable matrix polynomials.
  	\end{enumerate}
  	Test your program with the following matrix
  	\begin{gather*}
  	  \mata =
  	  \begin{pmatrix}
  	  1+i & 1 & 0 & 0 & 0\\
  	  0 & 1+i & 1 & 0 & 0\\
  	  0 & 0 & 1+i & 1 & 0\\
  	  0 & 0 & 0 & 1+i & 1\\
  	  1 & 0 & 0 & 0 & 1+i
  	  \end{pmatrix}.
  	\end{gather*}
  	
  \end{Problem}

  \begin{Problem}
	Let $\mata$ be a symmetric tridiagonal matrix. Show that the
	QR-iteration (see Algorithm 2.4.2 in the lecture notes) preserves
	the tridiagonal structure of the matrix, i.e., all iterates
	$\mata^{(n)}$ generated by the QR-iteration are tridiagonal.
\end{Problem}


\begin{Problem}
	Rewrite the QR factorization of a tridiagonal (complex) symmetric
	matrix such that its complexity is of order $O(n)$ (this proves
	the second part of Corollary 2.4.17 in the lecture notes).
\end{Problem}

\end{Sheet}


%%% Local Variables: 
%%% mode: latex
%%% TeX-master: "main"
%%% End: 
