%%%%%%%%%%%%%%%%%%%%%%%%%%%%%%%%%%%%%%
% Numerical Linear Algebra class 2022 
% Sheet 9
%%%%%%%%%%%%%%%%%%%%%%%%%%%%%%%%%%%%%%

\begin{Sheet}[discussion in the first tutorials of January, 2023]
  \label{sheet9}


  This exercise sheet reviews some topics of this lecture. Your
  answers do not have to be handed in, but the exercises will be
  discussed in the tutorials of the first week of January.

  \begin{Problem}
    Recapitulate the concept of an orthogonal projection and an
    oblique projection. What are use cases of both and why are they
    important for numerical linear algebra?
  \end{Problem}

  \begin{Problem}
    What methods presented in the lecture can be used for computing or
    estimating the eigenvalues of a matrix $\mata$? Sort by properties
    of the methods, as well as by conditions on $\mata$.
  \end{Problem}

  \begin{Problem}
    How can the QR factorization of a given matrix be computed?
    Discuss downsides and benefits of the different methods.
  \end{Problem}

  \begin{Problem}
    Householder reflections vs. Givens rotations: which one is more
    cost-efficient in the general case? When using the other one is
    advantageous?
  \end{Problem}

  \begin{Problem}
    Which of the discussed methods for solving eigenvalue problems can
    be implemented without explicitly forming a matrix?
  \end{Problem}

  \begin{Problem}
    Consider an $n\times n$ matrix that has $n$ distinct eigenvalues
    such that $\abs{\lambda_i} \neq \abs{\lambda_j}$ for $i\neq
    j$. How can the eigenvalue with the second largest absolute value
    be computed?
  \end{Problem}

  \begin{Problem}
    When does one step of the Gauss-Seidel iteration provide the
    direct solution of a linear system? Consider an upper triangular
    matrix to visualize this matter.
  \end{Problem}

\end{Sheet}


%%% Local Variables: 
%%% mode: latex
%%% TeX-master: "main"
%%% End: 
