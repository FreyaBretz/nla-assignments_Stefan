%%%%%%%%%%%%%%%%%%%%%%%%%%%%%%%%%%%%%%
% Numerical Linear Algebra class 2022 
% Sheet 7
%%%%%%%%%%%%%%%%%%%%%%%%%%%%%%%%%%%%%%

\begin{Sheet}[to be handed in until December 9, 2022, 2pm.]
  \label{sheet7}

  \begin{Problem}
    Let $\mata$ be a real symmetric matrix with eigenvalues
    $\lambda_j$, $1\leq j\leq n$. Show that for any polynomial
    $p(t) = \sum_{j=0}^k a_jt^j$
    the matrix
    \begin{gather*}
      p(\mata) = \sum_{j=0}^k a_j\mata^j
    \end{gather*}
    is also symmetric. Moreover, the eigenvectors of $\mata$ and
    $p(\mata)$ are identical, while the eigenvalues of $p(\mata)$ are
    equal to $p(\lambda_j)$ for $1\leq j\leq n$.
  \end{Problem}

  \begin{Problem}[Eigenvalues and eigenvectors of a tensor product
      operator]
     \label{sheet7:problem2}

    For vectors $\vu\in\R^n$ and $\vv\in\R^m$ we define the tensor
    product as a (block) vector
    \begin{gather*}
      \vu\otimes\vv =
      \begin{pmatrix}
        u_1v& u_2v& \cdots & u_nv
      \end{pmatrix}^T \in\R^{nm}.
    \end{gather*}
    Let $\mata$ and $\matb$ be matrices in $\R^{n\times n}$ and
    $\R^{m\times m}$ respectively. We define their tensor product
    $\mata\otimes\matb\in\R^{nm\times nm}$ as
    \begin{gather*}
      \mata\otimes\matb =
      \begin{pmatrix}
        a_{11}\matb & \cdots & a_{1n}\matb \\
        \vdots & \ddots & \vdots \\
        a_{n1}\matb & \cdots & a_{nn}\matb
      \end{pmatrix}.
    \end{gather*}
    \begin{enumerate}[(a)]
    \item Show that
      $(\mata\otimes\matb)(\vu\otimes\vv) =
      (\mata\vu)\otimes(\matb\vv)$.
    \item Prove that the eigenvectors of the tensor product are the
      pairwise tensor products of the eigenvectos of the individual
      matrices. What are the corresponding eigenvalues?
    \end{enumerate}
  \end{Problem}

  \begin{Problem}
    \label{sheet7:problem3}
    Let $\matl_d$ be the discretization of the $d$-dimensional Laplace
    operator on the unit square by the five-point stencel with a
    uniform Dirichlet boundary condition to the mesh size
    $h=\frac1{n+1}$.  In 1D it is given in terms of the matrix
    $\matt_2$ defined in \cref{sheet6:problem2} of \cref{sheet6} by
    \begin{gather*}
      \matl_1 = \frac1{h^2}\matt_2 \in\R^{n\times n}.
    \end{gather*}
    In 2D it is given by
    \begin{gather*}
      \matl_2 = \frac1{h^2}
      \begin{pmatrix}
        \matd &   -\id & & & \\
         -\id &  \matd & -\id   & & \\
              & \ddots & \ddots & \ddots & \\
              &        &   -\id &  \matd &   -\id\\
              &        &        &   -\id &  \matd
      \end{pmatrix}
      \in\R^{n^2\times n^2},
    \end{gather*}
    where $\id\in\R^{n\times n}$ and $\matd\in\R^{n\times n}$ are
    defined as
    \begin{gather*}
      \id =
      \begin{pmatrix}
        1 & & \\
          & \ddots & \\
          &        & 1
      \end{pmatrix}
      \quad\text{and}\quad
      \matd =
      \begin{pmatrix}
         4 & -1 &    && \\
        -1 &  4 & -1 && \\
           & \ddots & \ddots & \ddots & \\
           &    & -1 & 4 & -1 \\
           &    &    & -1 & 4
      \end{pmatrix}.
    \end{gather*}
    \begin{enumerate}[(a)]
    \item Use the results of \cref{sheet7:problem2} to show that
      $\matl_2$ could be expressed as
      \begin{gather*}
        \matl_2 = (\matl_1\otimes\id) + (\id\otimes \matl_1).
      \end{gather*}
    \item List the four smallest eigenvalues of $\matl_2$. What is
      their multiplicity?
    \end{enumerate}
  \end{Problem}

  \begin{Problem}[Programming]
    Write a program that solves the 2D Laplace problem
    \begin{gather*}
      \mata\vx=\vb,
    \end{gather*}
    with the Gauss-Seidel iteration, where the system matrix
    $\mata=\matl_2\in\R^{n^2\times n^2}$ is defined in
    \cref{sheet7:problem3}, by observing the following steps. Try to
    avoid explicitly forming the system matrix.
    \begin{enumerate}[(a)]
    \item\label{sheet7:problem4:parta} Implement a function
      \lstinline{vmult} that performs the matrix-vector product
      $\mata\cdot\vv$ of $\mata$ with a given vector $\vv$, and
      returns a vector $\vw$.
    \item Write a function \lstinline{gauss_seidel_step} that
      implements a single step of the Gauss-Seidel iteration. As in
      \eqref{sheet7:problem4:parta} this function should take a vector
      $\vv$ and return the resulting vector $\vw$.
    \item Use the null-vector $\vx^{(0)} = (0,...,0)^T$ as initial
      guess and test your program with $n=20$ and $n=100$. Observe the
      evolution of the residual in the 2-norm
      \begin{gather*}
        r^{(k)} = \norm{\mata\cdot\vx^{(k)}-\vb}_2
      \end{gather*}
      for 50 steps of the Gauss-Seidel iteration.  Plot the obtained
      result vector after 1, 5, 15, 30, and 50 iterations.

      \textit{Note: Be careful not to crash your computer when
        starting your program with $n=100$ or more, if you explicitly
        build the system matrix, because of the possibly high memory
        usage.}
    \end{enumerate}
  \end{Problem}

\end{Sheet}


%%% Local Variables: 
%%% mode: latex
%%% TeX-master: "main"
%%% End: 
