%%%%%%%%%%%%%%%%%%%%%%%%%%%%%%%%%%%%%%
% Numerical Linear Algebra class 2022 
% Sheet 3                             
%%%%%%%%%%%%%%%%%%%%%%%%%%%%%%%%%%%%%%

\begin{Sheet}[to be handed in until November 11, 2022, 2pm.]

  \begin{Problem}
    Consider a Hermitian matrix $\mata\in\C^{n\times n}$ and a unitary
    linear operator $\matq\in\C^m\to\C^n$, $m<n$. Prove, that an
    eigenvalue $\lambda_k(\matb)$ of the matrix
    $\matb=\matq^*\mata\matq\in\C^{m\times m}$ is either equal to 0,
    or the following estimate holds
    \begin{gather*}
      \abs{\lambda_{\min}(\mata)}
      \leq \abs{\lambda_{k}(\matb)}
      \leq \abs{\lambda_{\max}(\mata)},
    \end{gather*}
    where $\lambda_{\min}(\mata)$ and $\lambda_{\max}(\mata)$ denote
    the smallest and largest eigenvalues of $\mata$ measured by their
    magnitude.
  \end{Problem}

  \begin{Problem}
    A diagonalizable real matrix $\mata$ has the following spectrum:
    \begin{gather*}
      \sigma(\mata) = \set{-2,\ 1-2i,\ 1+2i,\ 1,\ -i,\ i,\ 2}.
    \end{gather*}
    Consider using the inverse power method (\textit{vector
      iteration}) to compute its eigenvalues.
    \begin{enumerate}[(a)]
    \item Find a set of all shift parameters for which the inverse
      power method may \textit{not} converge. Draw a sketch.
    \item For every \textit{real} eigenvalue find a \textit{real}
      range of shifts that, if used in the inverse power method, will
      reduce the error of approximation of the eigenvalue by a factor
      of $10$ in each iteration.
    \end{enumerate}
  \end{Problem}

  \begin{Problem}
    \label{problem:shift}
    Propose shift parameters that will allow you to compute
    \textit{all} eigenvalues and eigenvectors of the matrix
    \begin{gather*}
      \begin{pmatrix}
        100 & 15 & 3 \\
         15 & 20 & 5 \\
          3 &  5 & 65
      \end{pmatrix}
      .
    \end{gather*}
    Prove that your choice is correct.  \textit{Hint:} Gershgorin
    circle theorem
  \end{Problem}

  \begin{Problem}[Programming]
    Write a program that computes all eigenvectors and eigenvalues of
    the matrix
    \begin{gather*}
      \mata_{\epsilon} =
      \begin{pmatrix}
        100& 15& 3& 0& 0& \epsilon \\
        15& 20& 5& 0& \epsilon & 0 \\
        3& 5& 65& \epsilon&  0& 0 \\
        0& 0& \epsilon& 110& 20& 5 \\
        0& \epsilon& 0& 20& 80& 4 \\
        \epsilon& 0& 0& 5& 4& 30
      \end{pmatrix}
    \end{gather*}
    % with $\epsilon=1$
    using the shifted (inverse) power method by observing the
    following steps:
    \begin{enumerate}[(a)]
    \item Consider the matrix $\mata_0$ with $\epsilon=0$ and examine
      the structure of the eigenvalue problem.
    \item Compute all eigenvalues and eigenvectors of $\mata_0$.
    \item Make an \textit{educated} guess of the eigenvalues of
      $\mata_1$ and try computing those.
    \item Would the strategy you proposed in
      Problem~\ref{problem:shift} allow you to compute all 6
      eigenvalues?
    \end{enumerate}
  \end{Problem}
\end{Sheet}


%%% Local Variables: 
%%% mode: latex
%%% TeX-master: "main"
%%% End: 
