%%%%%%%%%%%%%%%%%%%%%%%%%%%%%%%%%%%%%%
% Numerical Linear Algebra class 2022 
% Sheet 4
%%%%%%%%%%%%%%%%%%%%%%%%%%%%%%%%%%%%%%

\begin{Sheet}[to be handed in until November 18, 2022, 2pm.]
  \label{sheet4}

  \begin{Problem}
    Let $\mata$ be a symmetric tridiagonal matrix. Show that the
    QR-iteration (see Algorithm 1.4.3 in the lecture notes) preserves
    the tridiagonal structure of the matrix, i.e., all iterates
    $\mata^{(n)}$ generated by the QR-iteration are tridiagonal.
  \end{Problem}

  \begin{Problem}
    Show that a (complex) symmetric matrix can be transformed to a
    tridiagonal matrix by using similarity transformations (this
    proves Theorem 1.4.14 in the lecture notes).
  \end{Problem}

  \begin{Problem}
    Rewrite the QR factorization of a tridiagonal (complex) symmetric
    matrix such that its complexity is of order $O(n)$ (this proves
    the second part of Corollary 1.4.13 in the lecture notes).
  \end{Problem}

  \begin{Problem}
    Find an example of a matrix with a real spectrum for which the QR
    method will \textit{not} converge to an upper triangular matrix.
  \end{Problem}

  \begin{Problem}[Programming]
    \hfill\\\vspace{-6ex}
    \begin{enumerate}[(a)]
    \item Implement the Hessenberg QR step (Algorithm 1.4.12 in the
      lecture notes) in real arithmetic.
    \item Test your code with the tridiagonal matrix
      $\mata_n=\operatorname{tridiag}(-1.,2.,-1.)$ in dimension $n=4$
      and check, if your results are correct.
    \item Use your implementation to run several steps of the QR
      iteration (Algorithm 1.4.3 in the lecture notes) for the matrix
      $\mata_{10}$.
    \item Discuss the observed convergence of the off-diagonal and
      diagonal entries, respectively.
    \end{enumerate}
  \end{Problem}

\end{Sheet}


%%% Local Variables: 
%%% mode: latex
%%% TeX-master: "main"
%%% End: 
