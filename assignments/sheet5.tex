%%%%%%%%%%%%%%%%%%%%%%%%%%%%%%%%%%%%%%
% Numerical Linear Algebra class 2022 
% Sheet 5
%%%%%%%%%%%%%%%%%%%%%%%%%%%%%%%%%%%%%%

\begin{Sheet}[to be handed in until November 25, 2022, 2pm.]
  \label{sheet5}

  \begin{Problem}
    Show that a normal triangular matrix is diagonal. \textit{Hint:}
    look at the norms of $\mata \ve_i$ and $\mata^*\ve_i$.
  \end{Problem}

  \begin{Problem}
    Prove that in case of a normal real matrix, for each complex
    eigenvalue pair there is a $2\times 2$ matrix with according
    invariant subspace.
    \begin{enumerate}[(a)]
    \item Show that complex eigenvalues of a real matrix come in
      complex conjugate pairs.
    \item Show that the eigenvectors are of the form $\vu\pm i\vv$.
    \item Choose real linear combinations of these vectors to obtain
      the $2\times 2$ block.
    \end{enumerate}
  \end{Problem}

  \begin{Problem}
    Provide the following steps of Lemma 1.5.11 in the lecture notes
    (explicit double shift).
    \begin{enumerate}[(a)]
    \item Show that $\matq_1\matq_2\matr_2\matr_1$ represents the QR
      factorization of a real matrix
      $\matm = (\matH - \sigma_1\id)(\matH - \sigma_2\id)$.
    \item Show that $\matq_1\matq_2$ is the orthogonal matrix that
      implements the similarity transformation of $\matH$ to obtain
      $\matH_2$.
    \end{enumerate}
  \end{Problem}

  \begin{Problem}[Programming]
    Implement the symmetric QR step with implicit shift (Algorithm
    1.5.6 in lecture notes) for a symmetric, unreduced, tridiagonal
    matrix $\matt$ by observing the following steps:
    \begin{enumerate}[(a)]
    \item Store $\matt$ in two vectors, one for the diagonal, and one
      for the subdiagonal entries.
    \item Implement the Givens rotation $\matg_{12}$ and think about
      where to store the additional non-zero entry $t_{31}$.
    \item Implement the additional Givens rotations for this data
      structure.
    \item Use this to compute the eigenvalues of the matrix
      $\mata_n=\operatorname{tridiag}(-1.,2.,-1.)$ in dimension
      $n=10$.
    \end{enumerate}
  \end{Problem}

\end{Sheet}


%%% Local Variables: 
%%% mode: latex
%%% TeX-master: "main"
%%% End: 
