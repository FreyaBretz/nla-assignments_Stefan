\begin{uebung}{Experimentelle Konvergenzraten}

Gegeben seien die Zahlenfolgen
\begin{center}
	\begin{tabular}{|r|r|r|r|}
        \hline $h$ & $a_h$ & $b_h$ & $c_h$ \\\hline
        1/2   & 1.07627    &  1.70051 & 0.429204    \\
        1/4   & 0.604185   &  1.71382 & 0.00455975  \\
        1/8   & 0.320317   &  1.71716 & 1.68691e-05 \\
        1/16  & 0.164945   &  1.71800 & 1.62880e-08 \\
        1/32  & 0.0836993  &  1.71821 & 3.96572e-12 \\
        1/64  & 0.0421601  &  1.71826 & 2.22045e-16 \\
        1/128 & 0.0211582  &  1.71828 & \\
        \hline
    \end{tabular}      
\end{center}

Die \textbf{Konvergenzordnung} einer Folge $x_h$ sei die größte
Zahl $\varrho$ so dass
$x_h = \mathcal O(h^\varrho)$ gilt. Sie kann berechnet werden als
    \begin{gather*}
      \varrho = \frac1{\log 2} \lim_{h\to 0} \log\left|\frac{x_h}{x_{\frac h2}}\right|.
    \end{gather*}        

\begin{enumerate}
	\item Bestimmen Sie eine Approximation der Konvergenzordnung $\varrho$ für $a_h$.
      Welche Zeilen der Tabelle benutzen sie dazu am besten? Wie
      verifizieren Sie Ihr Ergebnis?
    \item Sei $b=\lim_{h\to0} b_h$. Bestimmen Sie ohne $b$ zu kennen
      die ``intrinsische'' Konvergenzordnung der Folge $b-b_h$. Nutzen
      Sie dazu die Darstellung $b-b_h = b-b_{h/2} + b_{h/2} - b_h$, um die Formel
      \begin{gather*}
        \varrho \approx \frac1{\log 2}
        \log \left|\frac{b_h - b_{\frac h2}}{b_{\frac h2} -b_{\frac h4}}\right|
      \end{gather*}
      zu rechtfertigen.
    \item Kommentieren Sie die Frage der Konvergenzordnung der Folge $c_h$
\end{enumerate}
\end{uebung}

\begin{solution}
	Zu \textbf{a}: Da sich $\varrho$ mit Hilfe des Limes $\lim_{h\to 0} \log\left|\frac{x_h}{x_{\frac h2}}\right|$
	berechnet, nimmt man die letzten beiden Zeilen zur Abschätzung:
	\begin{equation}
		\varrho \approx \frac{1}{\log 2} \log \left|\frac{a_{1/64}}{a_{1/128}}\right| \approx 0.994661
	\end{equation}
	Es liegt also die Vermutung nahe, dass $\varrho = 1$.
	Betrachtet man die Werte von $\frac{1}{\log 2} \log \left|\frac{a_{h}}{a_{h/2}}\right|$ für andere Paare $(a_h,a_{h/2})$ dann nähern sie sich der 1 mit kleiner werdendem $h$.
	
	Zu \textbf{b}:
	Wir nehmen an, dass gilt $\vert b-b_h \vert=\mathcal{O}(h^{\varrho})=ch^{\varrho}$.
	Dann gilt:
	\begin{equation}
		\vert b_h - b_{h/2}\vert \leq \vert b_h- b\vert + \vert b-b_{h/2}\vert \approx ch^{\varrho} + c(h/2)^{\varrho}
	\end{equation}
	und 
	\begin{equation}
	\vert b_{h/2} - b_{h/4}\vert \leq \vert b_{h/2}- b\vert + \vert b-b_{h/4}\vert \approx c(h/2)^{\varrho} + c(h/4)^{\varrho}
	\end{equation}
	
	Wir gehen davon aus, dass die Konstante in allen drei Fällen (ungefähr) identisch ist.
	
	Dann können wir heuristisch folgern:
	
	\begin{equation}
		\frac{\vert b_h - b_{h/2}\vert}{\vert b_{h/2} - b_{h/4}\vert} \approx \frac{ch^{\varrho} + c(h/2)^{\varrho}}{c(h/2)^{\varrho}+ c(h/4)^{\varrho}} = 2^{\varrho}
	\end{equation}
	
	Es gilt also 
	\begin{equation}
		\frac1{\log 2}
		\log \left|\frac{b_h - b_{\frac h2}}{b_{\frac h2} -b_{\frac h4}}\right| \approx 
		\frac1{\log 2}\log 2^{\varrho} = \varrho.
	\end{equation}
	
	
	Zu \textbf{c}: Die Abschätzung für $\varrho$ mit obiger Formel steigt in jedem Schritt um die Konstante 2. Man kann also davon ausgehen, dass $c_h$ nicht wie $\mathcal{O}(h^{\varrho})$ konvergiert, sondern schneller?
	

\end{solution}