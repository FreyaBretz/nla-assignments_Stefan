%%%%%%%%%%%%%%%%%%%%%%%%%%%%%%%%%%%%%%
% Numerical Linear Algebra class 2022 
% Sheet 5
%%%%%%%%%%%%%%%%%%%%%%%%%%%%%%%%%%%%%%

\begin{Sheet}[to be handed in until November 22, 2023, 11am]
  \label{sheet5}

  \begin{Problem}
  	Problem 2.4.16 in the Lecture Notes:
  	\begin{enumerate}
  		\item How many operations do the two versions of the Hessenberg QR step require?
  		\item Show that if $\matH$ is Hermitian, the result of the
  		Hessenberg QR step is Hermitian as well.
  	\end{enumerate}
  \end{Problem}

  \begin{Problem}
	Problem 2.4.20 in the Lecture Notes:\\
	Show that every (complex) Hermitian matrix is orthogonally similar
	to a symmetric tridiagonal matrix with real entries.
  \end{Problem}

\begin{Problem}
	\hfill\\\vspace{-6ex}
	\begin{enumerate}[(a)]
		\item Implement the implicit Hessenberg QR step (Algorithm 2.4.9 in the
		lecture notes) in real arithmetic.
		\item Test your code with the tridiagonal matrix
		$\mata_n=\operatorname{tridiag}(-1.,2.,-1.)$ in dimension $n=4$.
		\item Use your implementation to run several steps of the QR
		iteration (Algorithm 2.4.19 in the lecture notes) for the matrix
		$\mata_{10}$.
		You should obtain the following eigenvalues:
		\begin{equation*}
		\lambda_j = 2 - 2\cos(j\theta) = 4\sin^2\left(\frac{j\theta}{2}\right)
		\end{equation*}
		where $\theta = \frac{\pi}{n+1}$, $1 \leq j \leq n$.
		\item Discuss the observed convergence of the off-diagonal and
		diagonal entries, respectively.
	\end{enumerate}
\end{Problem}

\begin{Problem}
	\hfill\\\vspace{-6ex}
	\begin{enumerate}[(a)]
		\item Implement the QR iteration with shift (Algorithm 2.4.25 in the Lecture Notes) using the Wilkinson shift (Definition 2.4.30).
		\item Test your implementation with the matrix $\mata_{10}$. Run several steps of the iteration and observe the behaviour of the subdiagonal elements, especially the last one.
		\item For the curious: Implement the QR iteration with deflation (cf. Algorithm 2.4.35).
	\end{enumerate} 		
\end{Problem}

\end{Sheet}


%%% Local Variables: 
%%% mode: latex
%%% TeX-master: "main"
%%% End: 
