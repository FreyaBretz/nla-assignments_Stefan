%%%%%%%%%%%%%%%%%%%%%%%%%%%%%%%%%%%%%%
% Numerical Linear Algebra class 2022 
% Sheet 3                             
%%%%%%%%%%%%%%%%%%%%%%%%%%%%%%%%%%%%%%

\begin{Sheet}[to be handed in until November 8, 2023, 11am.]
  \label{sheet3}
   
  \begin{Problem}
    Consider a Hermitian matrix $\mata\in\C^{n\times n}$ and a unitary
    linear operator $\matq\in\C^m\to\C^n$, $m<n$. Prove, that an
    eigenvalue $\lambda_k(\matb)$ of the matrix
    $\matb=\matq^*\mata\matq\in\C^{m\times m}$ is either equal to 0,
    or the following estimate holds
    \begin{gather*}
      \abs{\lambda_{\min}(\mata)}
      \leq \abs{\lambda_{k}(\matb)}
      \leq \abs{\lambda_{\max}(\mata)},
    \end{gather*}
    where $\lambda_{\min}(\mata)$ and $\lambda_{\max}(\mata)$ denote
    the smallest and largest eigenvalues of $\mata$ measured by their
    magnitude.
  \end{Problem}

  \begin{Problem}
  	Show that a normal triangular matrix is diagonal. \textit{Hint:}
  	look at the norms of $\mata \ve_i$ and $\mata^*\ve_i$.
  \end{Problem}


  \begin{Problem}
  	Consider the matrix $M = \begin{pmatrix}
  	\eta & 1\\
  	\eta &\eta
  	\end{pmatrix}$ with $|\eta| << 1$.\\
  	Explain why the problem of finding eigenvectors is \textit{not} well-posed in this example.
  \end{Problem}

  \begin{Problem}
  	Consider the following matrix
  	\begin{gather*}
  	\mata =
  	\begin{pmatrix}
  	\cos\phi & -\sin\phi\\
  	\sin\phi &  \cos\phi
  	\end{pmatrix}^T
  	\begin{pmatrix}
  	1 & \\
  	& c
  	\end{pmatrix}
  	\begin{pmatrix}
  	\cos\phi & -\sin\phi\\
  	\sin\phi &  \cos\phi
  	\end{pmatrix}
  	\end{gather*}
  	with parameters $\phi\in[0,2\pi]$ and $c\in(0,1)$.
  	\begin{enumerate}
  		\item Compute the eigenvalues and eigenvectors of $\mata$.
  		\item (Programming) Write a program which computes the sequence
  		$\vx^{(n)}\in\R^2$ defined as
  		\begin{align*}
  		\vx^{(n)} &= \mata \vx^{(n-1)}, \\
  		\vx^{(0)} &= \vx^{*},
  		\end{align*}
  		for $\vx^{*} = (1,\ 0)^T$, $c = 0.1$, and
  		$\phi=\frac\pi4$. Try playing with different values of those
  		parameters.
  		\item Is there a limit of $\vx^{(n)}$? What is about the case
  		$c=1$?
  		\item Compute the limit: $\lim_{n\to\infty}\mata^n$.
  	\end{enumerate}
  \end{Problem}

\end{Sheet}


%%% Local Variables: 
%%% mode: latex
%%% TeX-master: "main"
%%% End: 