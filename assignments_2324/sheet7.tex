%%%%%%%%%%%%%%%%%%%%%%%%%%%%%%%%%%%%%%
% Numerical Linear Algebra class 2022 
% Sheet 7
%%%%%%%%%%%%%%%%%%%%%%%%%%%%%%%%%%%%%%

\begin{Sheet}[to be handed in until December 6, 2023, 11am.]
  \label{sheet7}

  \begin{Problem}
    Prove Theorem 3.2.15 in the Lecture Notes (Convergence of Richardson Iteration).
  \end{Problem}

  \begin{Problem}
  	Gegeben seien die Zahlenfolgen
  	\begin{center}
  		\begin{tabular}{|r|r|r|r|}
  			\hline $h$ & $a_h$ & $b_h$ & $c_h$ \\\hline
  			1/2   & 1.07627    &  1.70051 & 0.429204    \\
  			1/4   & 0.604185   &  1.71382 & 0.00455975  \\
  			1/8   & 0.320317   &  1.71716 & 1.68691e-05 \\
  			1/16  & 0.164945   &  1.71800 & 1.62880e-08 \\
  			1/32  & 0.0836993  &  1.71821 & 3.96572e-12 \\
  			1/64  & 0.0421601  &  1.71826 & 2.22045e-16 \\
  			1/128 & 0.0211582  &  1.71828 & \\
  			\hline
  		\end{tabular}      
  	\end{center}
  	
  	Die \textbf{Konvergenzordnung} einer Folge $x_h$ sei die größte
  	Zahl $\varrho$ so dass
  	$x_h = \mathcal O(h^\varrho)$ gilt. Sie kann berechnet werden als
  	\begin{gather*}
  	\varrho = \frac1{\log 2} \lim_{h\to 0} \log\left|\frac{x_h}{x_{\frac h2}}\right|.
  	\end{gather*}        
  	
  	\begin{enumerate}
  		\item Bestimmen Sie eine Approximation der Konvergenzordnung $\varrho$ für $a_h$.
  		Welche Zeilen der Tabelle benutzen sie dazu am besten? Wie
  		verifizieren Sie Ihr Ergebnis?
  		\item Sei $b=\lim_{h\to0} b_h$. Bestimmen Sie ohne $b$ zu kennen
  		die ``intrinsische'' Konvergenzordnung der Folge $b-b_h$. Nutzen
  		Sie dazu die Darstellung $b-b_h = b-b_{h/2} + b_{h/2} - b_h$, um die Formel
  		\begin{gather*}
  		\varrho \approx \frac1{\log 2}
  		\log \left|\frac{b_h - b_{\frac h2}}{b_{\frac h2} -b_{\frac h4}}\right|
  		\end{gather*}
  		zu rechtfertigen.
  		\item Kommentieren Sie die Frage der Konvergenzordnung der Folge $c_h$
  	\end{enumerate}
  \end{Problem}

  \begin{Problem}[Programming]
    \label{sheet7:problem3}
    Let $\matl_d$ be the discretization of the $d$-dimensional Laplace
    operator on the unit square by the five-point stencel with a
    uniform Dirichlet boundary condition to the mesh size
    $h=\frac1{n+1}$.  In 1D it is given in terms of the matrix
    $\matt_2$ defined in \cref{sheet6:problem4} of \cref{sheet6} by
    \begin{gather*}
    \matl_1 = \frac1{h^2}\matt_2 \in\R^{n\times n}.
    \end{gather*}
    In 2D it is given by
    \begin{gather*}
    \matl_2 = \frac1{h^2}
    \begin{pmatrix}
    \matd &   -\id & & & \\
    -\id &  \matd & -\id   & & \\
    & \ddots & \ddots & \ddots & \\
    &        &   -\id &  \matd &   -\id\\
    &        &        &   -\id &  \matd
    \end{pmatrix}
    \in\R^{n^2\times n^2},
    \end{gather*}
    where $\id\in\R^{n\times n}$ and $\matd\in\R^{n\times n}$ are
    defined as
    \begin{gather*}
    \id =
    \begin{pmatrix}
    1 & & \\
    & \ddots & \\
    &        & 1
    \end{pmatrix}
    \quad\text{and}\quad
    \matd =
    \begin{pmatrix}
    4 & -1 &    && \\
    -1 &  4 & -1 && \\
    & \ddots & \ddots & \ddots & \\
    &    & -1 & 4 & -1 \\
    &    &    & -1 & 4
    \end{pmatrix}.
    \end{gather*}
    
    Let $\mata=\matl_2\in\R^{n^2\times n^2}$ as defined above and $\vb=(1,...,1)^T$.
    Write a program that solves the 2D Laplace problem
    \begin{gather*}
      \mata\vx=\vb,
    \end{gather*}
    with the Richardson iteration by observing the following
    steps:
    \begin{enumerate}[(a)]
    \item\label{sheet7:problem4:parta} Implement a function
      \lstinline{vmult} that performs the matrix-vector product
      $\mata\cdot\vv$ of $\mata$ with a given vector $\vv$, and
      returns a vector $\vw$. 
      Do not explicitly form the system matrix in this function.
      The parameters should only be the vector $\vv$ and, optionally, $n$.
    \item Write a function \lstinline{richardson_step} that
      implements a single step of the Richardson iteration. As in
      \eqref{sheet7:problem4:parta} this function should take a vector
      $\vv$, optionally $n$, and return the resulting vector $\vw$.
      What is the optimal choice for the relaxation parameter $\omega_k$?
      The eigenvalues of $\matl_2$ are given by the pairwise sums of the eigenvalues of $\matt_2$.
    \item Use the null-vector $\vx^{(0)} = (0,...,0)^T$ as initial
      guess and test your program with $n=20$ and $n=100$.
      Observe the evolution of the residual in the 2-norm
      \begin{gather*}
        r^{(k)} = \norm{\mata\cdot\vx^{(k)}-\vb}_2
      \end{gather*}
      for 50 steps of the Richardson iteration.  Plot the obtained
      result vector after 1, 5, 15, 30, and 50 iterations.
      \textit{Hint: For plotting you have to invert the numbering of the vector (use, for example, \lstinline{np.reshape}) and then plot the two dimensional grid using a numpy color map (greyscale).
      	Details may be found in appendix B.3 on finite difference methods in the lecture notes. }      
    \end{enumerate}
  \end{Problem}

\end{Sheet}


%%% Local Variables: 
%%% mode: latex
%%% TeX-master: "main"
%%% End: 
