%%%%%%%%%%%%%%%%%%%%%%%%%%%%%%%%%%%%%%
% Numerical Linear Algebra class 2022 
% Sheet 3                             
%%%%%%%%%%%%%%%%%%%%%%%%%%%%%%%%%%%%%%

\begin{Sheet}[to be handed in until November 8, 2023, 11am.]
  \label{sheet3}
  
  \textbf{Normal and Hermitian Matrices}
  
  \begin{Problem}
    Consider a Hermitian matrix $\mata\in\C^{n\times n}$ and a unitary
    linear operator $\matq\in\C^m\to\C^n$, $m<n$. Prove, that an
    eigenvalue $\lambda_k(\matb)$ of the matrix
    $\matb=\matq^*\mata\matq\in\C^{m\times m}$ is either equal to 0,
    or the following estimate holds
    \begin{gather*}
      \abs{\lambda_{\min}(\mata)}
      \leq \abs{\lambda_{k}(\matb)}
      \leq \abs{\lambda_{\max}(\mata)},
    \end{gather*}
    where $\lambda_{\min}(\mata)$ and $\lambda_{\max}(\mata)$ denote
    the smallest and largest eigenvalues of $\mata$ measured by their
    magnitude.
  \end{Problem}

  \begin{Problem}
  	Show that a normal triangular matrix is diagonal. \textit{Hint:}
  	look at the norms of $\mata \ve_i$ and $\mata^*\ve_i$.
  \end{Problem}

  \textbf{Well-posedness of EVP}

  \begin{Problem}
  	Construct a counterexample that the problem of finding
  	eigenvectors is \textit{not} well-posed, if the eigenspaces are
  	almost parallel.
  \end{Problem}

  \begin{Problem}
  	Consider the following matrix
  	\begin{gather*}
  	\mata =
  	\begin{pmatrix}
  	\cos\phi & -\sin\phi\\
  	\sin\phi &  \cos\phi
  	\end{pmatrix}^T
  	\begin{pmatrix}
  	1 & \\
  	& c
  	\end{pmatrix}
  	\begin{pmatrix}
  	\cos\phi & -\sin\phi\\
  	\sin\phi &  \cos\phi
  	\end{pmatrix}
  	\end{gather*}
  	with parameters $\phi\in[0,2\pi]$ and $c\in(0,1)$.
  	\begin{enumerate}
  		\item Compute the eigenvalues and eigenvectors of $\mata$.
  		\item (Programming) Write a program which computes the sequence
  		$\vx^{(n)}\in\R^2$ defined as
  		\begin{align*}
  		\vx^{(n)} &= \mata \vx^{(n-1)}, \\
  		\vx^{(0)} &= \vx^{*},
  		\end{align*}
  		for $\vx^{*} = (1,\ 0)^T$, $c = 0.1$, and
  		$\phi=\frac\pi4$. Try playing with different values of those
  		parameters.
  		\item Is there a limit of $\vx^{(n)}$? What is about the case
  		$c=1$?
  		\item Compute the limit: $\lim_{n\to\infty}\mata^n$.
  	\end{enumerate}
  \end{Problem}

  \textbf{Vector Iterations: Simple Iterations}

  \begin{Problem}
    A diagonalizable real matrix $\mata$ has the following spectrum:
    \begin{gather*}
      \sigma(\mata) = \set{-2,\ 1-2i,\ 1+2i,\ 1,\ -i,\ i,\ 2}.
    \end{gather*}
    Consider using the inverse power method (\textit{vector
      iteration}) to compute its eigenvalues.
    \begin{enumerate}[(a)]
    \item Find a set of all shift parameters for which the inverse
      power method may \textit{not} converge. Draw a sketch.
    \item For every \textit{real} eigenvalue find a \textit{real}
      range of shifts that, if used in the inverse power method, will
      reduce the error of approximation of the eigenvalue by a factor
      of $10$ in each iteration.
    \end{enumerate}
  \end{Problem}

  \begin{Problem}
    \label{problem:shift}
    Propose shift parameters that will allow you to compute
    \textit{all} eigenvalues and eigenvectors of the matrix
    \begin{gather*}
      \begin{pmatrix}
        100 & 15 & 3 \\
         15 & 20 & 5 \\
          3 &  5 & 65
      \end{pmatrix}
      .
    \end{gather*}
    Prove that your choice is correct.  \textit{Hint:} Gershgorin
    circle theorem
  \end{Problem}

  \begin{Problem}[Programming]
    Write a program that computes all eigenvectors and eigenvalues of
    the matrix
    \begin{gather*}
      \mata_{\epsilon} =
      \begin{pmatrix}
        100& 15& 3& 0& 0& \epsilon \\
        15& 20& 5& 0& \epsilon & 0 \\
        3& 5& 65& \epsilon&  0& 0 \\
        0& 0& \epsilon& 110& 20& 5 \\
        0& \epsilon& 0& 20& 80& 4 \\
        \epsilon& 0& 0& 5& 4& 30
      \end{pmatrix}
    \end{gather*}
    using the shifted (inverse) power method by observing the
    following steps:
    \begin{enumerate}[(a)]
    \item\label{sheet3:prob4:parta} Consider the matrix $\mata_0$ with $\epsilon=0$ and examine
      the structure of the eigenvalue problem.
    \item Compute all eigenvalues and eigenvectors of $\mata_0$.
    \item Does the algorithm work, if you use Gerschgorin for the computation of all six eigenvalues?
    \item Can you use part \eqref{sheet3:prob4:parta} to get something
        better?
    \end{enumerate}
  \end{Problem}
\end{Sheet}


%%% Local Variables: 
%%% mode: latex
%%% TeX-master: "main"
%%% End: 
